% ==============================================================================
% Beamer Preamble — Introduction to Econometrics
% CREST, Institut Polytechnique de Paris
% ==============================================================================

% --- Document Class ---
% (loaded by each lecture file: \documentclass[aspectratio=169,12pt]{beamer})

% --- Theme ---
\usetheme{Madrid}
\usecolortheme{default}
\usefonttheme{professionalfonts}

% Remove navigation symbols
\setbeamertemplate{navigation symbols}{}

% Slide numbers in footer
\setbeamertemplate{footline}{%
  \hfill\insertframenumber/\inserttotalframenumber\hspace{0.5em}\vspace{0.5em}%
}

% ==============================================================================
% COLOR DEFINITIONS — IP Paris / CREST Palette
% ==============================================================================

\usepackage{xcolor}

\definecolor{ippblue}{HTML}{233E5C}       % IP Paris primary dark blue
\definecolor{crestblue}{HTML}{144563}      % CREST blue
\definecolor{accentblue}{HTML}{018BD3}     % IP Paris vibrant blue
\definecolor{lightblue}{HTML}{61ABF6}      % Light blue
\definecolor{darknavy}{HTML}{011A29}       % CREST dark navy
\definecolor{warmaccent}{HTML}{C0392B}     % Red accent for emphasis

% Semantic aliases
\definecolor{positive}{HTML}{15803D}       % Green — good/identified
\definecolor{negative}{HTML}{B91C1C}       % Red — bad/problematic
\definecolor{muted}{HTML}{525252}          % Gray — reference/context

% Apply to Beamer elements
\setbeamercolor{structure}{fg=ippblue}
\setbeamercolor{title}{fg=ippblue}
\setbeamercolor{frametitle}{fg=ippblue,bg=white}
\setbeamercolor{title in head/foot}{fg=white,bg=ippblue}
\setbeamercolor{author in head/foot}{fg=white,bg=crestblue}
\setbeamercolor{date in head/foot}{fg=white,bg=accentblue}
\setbeamercolor{item}{fg=ippblue}
\setbeamercolor{subitem}{fg=crestblue}
\setbeamercolor{block title}{fg=white,bg=ippblue}
\setbeamercolor{block body}{bg=ippblue!5}
\setbeamercolor{block title alerted}{fg=white,bg=warmaccent}
\setbeamercolor{block body alerted}{bg=warmaccent!5}
\setbeamercolor{block title example}{fg=white,bg=positive}
\setbeamercolor{block body example}{bg=positive!5}

% ==============================================================================
% FONTS (XeLaTeX)
% ==============================================================================

\usepackage{fontspec}
\setsansfont{Source Sans Pro}[
  UprightFont = *-Regular,
  BoldFont = *-Bold,
  ItalicFont = *-Italic,
  BoldItalicFont = *-BoldItalic,
  Scale = 1.0
]

% ==============================================================================
% MATH PACKAGES
% ==============================================================================

\usepackage{amsmath}
\usepackage{amssymb}
\usepackage{mathtools}
\usepackage{bm}

% ==============================================================================
% MATH OPERATORS (Econometrics)
% ==============================================================================

\DeclareMathOperator{\E}{\mathbb{E}}
\DeclareMathOperator{\Var}{Var}
\DeclareMathOperator{\Cov}{Cov}
\DeclareMathOperator{\Corr}{Corr}
\DeclareMathOperator{\plim}{plim}
\DeclareMathOperator{\rank}{rank}
\DeclareMathOperator{\tr}{tr}
\newcommand{\indep}{\perp\!\!\!\perp}
\newcommand{\iid}{\overset{\text{iid}}{\sim}}
\newcommand{\convd}{\xrightarrow{d}}
\newcommand{\convp}{\xrightarrow{p}}
\newcommand{\convas}{\xrightarrow{a.s.}}

% ==============================================================================
% OTHER PACKAGES
% ==============================================================================

\usepackage{graphicx}
\usepackage{booktabs}
\usepackage{tikz}
\usetikzlibrary{arrows.meta,positioning,calc,decorations.pathreplacing}
\usepackage{hyperref}
\hypersetup{
  colorlinks=true,
  linkcolor=ippblue,
  urlcolor=accentblue,
  citecolor=crestblue
}

% Bibliography
\usepackage[round,authoryear]{natbib}

% ==============================================================================
% TEXT COMMANDS
% ==============================================================================

\newcommand{\key}[1]{\textcolor{warmaccent}{\textbf{#1}}}
\newcommand{\highlight}[1]{\textcolor{accentblue}{#1}}
\newcommand{\mutedtext}[1]{\textcolor{muted}{#1}}

% tcolorbox is required for custom environments
\usepackage{tcolorbox}
\tcbuselibrary{skins,breakable}

% ==============================================================================
% CUSTOM ENVIRONMENTS
% ==============================================================================

% --- keybox: Red accent background for key takeaways ---
\newenvironment{keybox}{%
  \begin{tcolorbox}[
    colback=warmaccent!8,
    colframe=warmaccent,
    leftrule=4pt,
    rightrule=0pt,
    toprule=0pt,
    bottomrule=0pt,
    arc=4pt,
    boxsep=4pt,
    left=8pt,
    right=8pt
  ]%
}{%
  \end{tcolorbox}%
}

% --- highlightbox: Light blue left-accent for highlights ---
\newenvironment{highlightbox}{%
  \begin{tcolorbox}[
    colback=accentblue!6,
    colframe=accentblue,
    leftrule=4pt,
    rightrule=0pt,
    toprule=0pt,
    bottomrule=0pt,
    arc=4pt,
    boxsep=4pt,
    left=8pt,
    right=8pt
  ]%
}{%
  \end{tcolorbox}%
}

% --- methodbox: Dark blue left-accent for methods ---
\newenvironment{methodbox}{%
  \begin{tcolorbox}[
    colback=ippblue!5,
    colframe=ippblue,
    leftrule=4pt,
    rightrule=0pt,
    toprule=0pt,
    bottomrule=0pt,
    arc=4pt,
    boxsep=4pt,
    left=8pt,
    right=8pt
  ]%
}{%
  \end{tcolorbox}%
}

% --- assumptionbox: Blue-bordered box for formal assumptions ---
\newenvironment{assumptionbox}{%
  \begin{tcolorbox}[
    colback=crestblue!4,
    colframe=crestblue,
    arc=4pt,
    boxsep=4pt,
    left=8pt,
    right=8pt
  ]%
}{%
  \end{tcolorbox}%
}

% --- definitionbox[Title]: Blue-bordered titled box ---
\newenvironment{definitionbox}[1][Definition]{%
  \begin{tcolorbox}[
    title=#1,
    colback=crestblue!4,
    colframe=crestblue,
    coltitle=white,
    fonttitle=\bfseries,
    arc=4pt,
    boxsep=4pt,
    left=8pt,
    right=8pt
  ]%
}{%
  \end{tcolorbox}%
}

% --- resultbox: Bold-bordered result box for theorems/propositions ---
\newenvironment{resultbox}{%
  \begin{tcolorbox}[
    colback=warmaccent!6,
    colframe=warmaccent,
    boxrule=2pt,
    arc=4pt,
    boxsep=4pt,
    left=8pt,
    right=8pt
  ]%
}{%
  \end{tcolorbox}%
}

% --- eqbox: Subtle blue background for featured equations ---
\newenvironment{eqbox}{%
  \begin{tcolorbox}[
    colback=ippblue!3,
    colframe=ippblue!3,
    boxrule=0pt,
    arc=0pt,
    boxsep=4pt,
    left=8pt,
    right=8pt
  ]%
}{%
  \end{tcolorbox}%
}
